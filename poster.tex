\documentclass[10pt, sans, mathserif]{beamer}
\usepackage{graphics}
\usepackage[utf8]{inputenc}
\usepackage[english]{babel}
\usepackage[size=custom,width=100,height=100,scale=1.2,orientation=portrait]{beamerposter}
\usepackage[absolute,overlay]{textpos}
\usepackage{multirow}
\usepackage{titlecaps}

%%%%%%%%%%%%%%%%%%%%%%%%%%%%%%%%%%%
% MATH
\usepackage{amsmath}
\usepackage{amssymb}
\usepackage{mathtools}

\DeclarePairedDelimiter\abs{\lvert}{\rvert}%

\newcommand{\sumsum}{
    \sum_{m = 0}^{M_s-1}\sum_{n = 0}^{N_s-1}
}
\newcommand{\sad}{
    \sumsum{\abs*{B_{s,p}(m,n) - C_{s,p,x,y}(m,n)}}
}

\newcommand{\minsad}{
    \ensuremath{\text{minSAD}}
}

\newcommand{\lowersad}{
    \text{SAD}^\Omega
}

%%%%%%%%%%%%%%%%%%%%%%%%%%%%%%%%%%%
% GLOSSARIES
\usepackage[acronym]{glossaries}
\loadglsentries{acronyms}

%%%%%%%%%%%%%%%%%%%%%%%%%%%%%%%%%%%
% TIKZ
\usepackage{tikz}
\usepackage{pgfplots}
\usepackage{tkz-euclide}


\usetikzlibrary{positioning}
\usetikzlibrary{shapes,arrows}
\usetikzlibrary{arrows.meta}

%%%%%%%%%%%%%%%%%%%%%%%%%%%%%%%%%%%
% REFERENCES
\usepackage{cleveref}

\crefname{figure}{Fig.}{Figures}

%%%%%%%%%%%%%%%%%%%%%%%%%%%%%%%%%%%
% BEAMER SETTINGS
\setbeamersize{text margin left=18cm,text margin right=2cm}
\setbeamerfont{normal text}{size=\normalsize}
\usepackage{sourcesanspro}
%\usefonttheme[onlymath]{mathpazo}
%\usepackage{kerkis} % Or palatino or mathpazo
%\usepackage{concmath}
%\usepackage{concrete}

%\usefonttheme{structurebold}
\setbeamertemplate{caption}[numbered]

%\def\baselinestretch{1.05}


\begin{document}

\usebackgroundtemplate{\includegraphics{icip.pdf}}

\begin{frame}[t]
    \centering

    %%%%%%%%%%%%%%%%%%%% HEADER %%%%%%%%%%%%%%%%%%%%%%%%%

    \huge{Sub-Partition Reuse For Fast Optimal Motion Estimation\\ In HEVC Successive Elimination Algorithms}

    \LARGE{Luc Trudeau, Stéphane Coulombe, Christian Desrosiers}

    \Large{Department of Software and IT Engineering, École de technologie supérieure, Montréal, Canada}

    %%%%%%%%%%%%%%%%%%%%%%%%% INTRO %%%%%%%%%%%%%%%%%%%%%%%%%
    \normalsize

 \begin{columns}[t, onlytextwidth]
    \begin{column}{0.5\textwidth}
        \begin{block}{\titlecap{1. Introduction}}
        %\begin{minipage}{0.55\textwidth}
            \begin{itemize}
            % Here is a problem
            \item Motion Estimation~(ME) is a crucial tool for video encoders.
            \item ME seeks the best candidate block~($C$) from a search area~($\mathcal{S}$) in a previously coded frame to predict the current block~($B$) (see \cref{fig:MotionEstimation}).
            % It's an interesting problem
            \item For HEVC, considering every candidate is prohibitively expensive, so modern search algorithms often find sub-optimal solutions.
            \item We want to reduce the number of candidates without sacrificing the optimal solution.
            % Here is my idea
            \item We propose an early termination scheme for square \glspl{pu} based on information reuse from rectangular ones.
            \end{itemize}
        %\end{minipage}\hfill
        % \begin{minipage}{0.45\textwidth}
            \begin{figure}[H]
                \vspace{-2em}
                \centering
                \begin{tikzpicture}[every node/.style={minimum size=1cm, font=\tiny},on grid, scale=1.1]
    \tikzstyle{every node}=[font=\normalsize]
    \begin{scope}[
	yshift=170,every node/.append style={
	    yslant=0.5,xslant=-1},yslant=0.5,xslant=-1
	             ]
        \fill[white,fill opacity=.9] (0,0) rectangle (5,5);
        \draw[black,very thick] (0,0) rectangle (5,5);
        \draw[step=5mm, black] (0,0) grid (5,5);
        \fill[red] (2.5,2.5) rectangle (3,3);
    \end{scope}

    \begin{scope}[
	yshift=0,every node/.append style={
	    yslant=0.5,xslant=-1},yslant=0.5,xslant=-1
	  ]
        \fill[white,fill opacity=0.6] (0,0) rectangle (5,5);
        \draw[black,very thick] (1.5,1.5) rectangle (4,4);
        \draw[black,ultra thick, dashed] (0,0) rectangle (5,5);
        \fill[green] (2,2) rectangle (2.5,2.5);
        \draw[->, ultra thick](2.75,2.75) -- (2.25, 2.25);
    \end{scope}

    \draw[-latex, ultra thick,dotted](5,9.5)node[right]{Current Frame}
        to[out=180,in=90] (4,8);
    \draw[-latex, ultra thick,dotted](5,10.75)node[right]{Current Block ($B$)}
        to[out=180,in=90] (0,8.9);

	\draw[-latex,ultra thick,dotted] (3,6) node[right]{$i$th Motion Vector $(x_i,y_i)$}
         to[out=180,in=90] (0,2.75);
	\draw[-latex,ultra thick,dotted] (4,5) node[right]{Search Area ($\mathcal{S}$)}
         to[out=180,in=90] (0.6,3.7);
    \draw[-latex,ultra thick,dotted] (4,4) node[right]{$i$th Block Matching} to[out=180,in=90] (0.2,2.2);
    \node at (8.5,3) {Candidate ($C$)};
    \draw[-latex,ultra thick,dotted] (3,1) node[right]{Previously Coded Frame}
         to[out=180,in=90] (0.5,0.3);

\end{tikzpicture}

                \vspace{-1em}
                \caption{Motion estimation finds the best candidate to predict the current block.}
                \label{fig:MotionEstimation}
            \end{figure}
        %\end{minipage}

        \end{block}

        %%%%%%%%%%%%%%%%%%%%%%%%% SEA %%%%%%%%%%%%%%%%%%%%%%%%%
            \begin{block}{\titlecap{2. Successive Elimination Algorithm}}
            \begin{itemize}
            \item Let $s \in \{\mathbb{S}, \mathbb{V}, \mathbb{H}\}$ be the partitioning shape of a \gls{pu} and $p$ be the partition index
            \begin{figure}[htb]
                \centering
                \begin{tikzpicture}[thick,scale=1.5]
\draw[] (0,0) rectangle (2,2);
\node[below] at (1,0) {$\mathbb{S}$};
\draw[] (2.5,0) rectangle (3.5,2);
\draw[] (3.5,0) rectangle (4.5,2);
\node[below] at (3.5,0) {$\mathbb{V}$};
\draw[] (5,0) rectangle (7,1);
\draw[] (5,1) rectangle (7,2);
\node[below] at (6,0) {$\mathbb{H}$};

\node at (1, 1) {$0$};
\node at (3, 1) {$0$};
\node at (4, 1) {$1$};
\node at (6, 1.5) {$0$};
\node at (6, 0.5) {$1$};
\end{tikzpicture}
                \caption{The first partition index is $0$ and if a second partition exists, its index is $1$.}
                \label{fig:CUShapes}
            \end{figure}
            \item The candidate at position $(x,y)$ is evaluated using
            \[
                \text{RCSAD}(s,p,x,y) = \sad + \lambda R(x,y) \:.
            \]

            \item Successive elimination uses a lower bound approximation of the RCSAD
            \[ \small
                \text{RCADS}(s,p,x,y) = \abs*{\sumsum{B}_{s,p}(m,n) - \sumsum{C_{s,p,x,y}(m,n)}} + \lambda R(x,y)\:.
                \label{eq:RCADS}
            \]

            \item Let $(\hat{x},\hat{y})$ be the position of the current best candidate. By transitivity:
            \begin{align*}
                \text{RCADS}(s,p,x,y) &\geqslant \text{RCSAD}(s,p,\hat{x},\hat{y}) \nonumber \\
                \implies \text{RCSAD}(s,p,x,y) &\geqslant \text{RCSAD}(s,p,\hat{x},\hat{y})\:.
                \label{eq:TransitiveElimination}
            \end{align*}
            \end{itemize}
        \end{block}


    %%%%%%%%%%%%%%%%%%%%%%%%% INFORMATION REUSE BETWEEN PU SHAPES %%%%%%%%%%%%%%%%%%%%%%%%%
        \begin{block}{\titlecap{3. Information reuse between PU shapes}}
           \begin{itemize}
           \item Traditionally, \glspl{pu} are evaluated  in the order \[\mathbb{S} \rightarrow \mathbb{V} \rightarrow \mathbb{H}\:.\]
           \item Consider the following orders \[\mathbb{V} \rightarrow \mathbb{H} \rightarrow \mathbb{S} \text{ and } \mathbb{H} \rightarrow \mathbb{V} \rightarrow \mathbb{S}\:,\] which allow for information reuse from $\mathbb{V}$ and/or $\mathbb{H}$ into $\mathbb{S}$. Such as
        \begin{equation}
            \lowersad =
            \max \begin{pmatrix}
                 \minsad(\mathbb{V},0) &+ \minsad(\mathbb{V},1), \nonumber \\
                 \minsad(\mathbb{H},0) &+ \minsad(\mathbb{H},1)
            \end{pmatrix}.
        \end{equation}
        \item It follows that
        \[
            \lowersad \leqslant \text{SAD}(\mathbb{S},0,x,y),~\forall~ (x,y) \in  \mathcal{S}_{\mathbb{S},0} \:.
        \]
                \item At worst, the min SAD of a partitioning is the min SAD of the block
        \begin{equation*} \text{SAD} \left(\: \tikz[scale=0.70,baseline=-0.25cm,very thick]{\draw (-2,-2) rectangle (2,2); \draw (0,-2) -- (0,2); \draw[->] (-1, 0) -- (-0.5, 0.5); \draw[->] (1, 0) -- (0.5, -0.5);
        }\:\right) \leqslant \text{SAD} \left(\: \tikz[scale=0.70,baseline=-0.25cm,very thick]{\draw (-2,-2) rectangle (2,2);  \draw[->] (0, 0) -- (0.5, -0.5); }\:\right)\:.
        \end{equation*}
        \end{itemize}
        \end{block}


                \begin{block}{\titlecap{4. Improved early termination for $\mathbb{S}$}}
            \begin{itemize}
                \item We evaluate candidates in increasing order of rate. When the rate is large the search can terminate (without evaluating the remainder of $\mathcal{S}$).
                \item Early termination rate proposed at ICIP 2014
                \[
                    R(x,y) \geqslant \frac{\text{SAD}(s,p,\hat{x},\hat{y})}{\lambda} + R(\hat{x},\hat{y}) \:.
                    \label{eq:Thres}
                \]
                \item Improved early termination rate for $\mathbb{S}$
                \[
                    R(x,y) \geqslant \frac{\text{SAD}(\mathbb{S},0,\hat{x},\hat{y}) - \lowersad}{\lambda} + R(\hat{x},\hat{y}) \:.
                    \label{eq:SThres}
                \]
            \end{itemize}
        \end{block}

    \end{column}
    \begin{column}{0.5\textwidth}

%%%%%%%%%%%%%%%%%%%%%%%%% EARLY TERMINATION %%%%%%%%%%%%%%%%%%%%%%%%%

        \begin{block}{}
        \begin{figure}
                \centering
                \vspace{-1em}
                \begin{tikzpicture}[scale=1.3]
                \tikzstyle{every node}=[font=\small]

                \draw[gray] (0,0) grid (20,16);
                \draw[line width=.5mm] (0,15) -- node[above] {$\lambda$} (15,0);

                \draw[decoration={markings,mark=at position 1 with
    {\arrow[scale=4,>=stealth]{>}}},postaction={decorate}, line width=.75mm] (0,0) -- (0,16) node[above] {Distortion};
                \draw[decoration={markings,mark=at position 1 with
    {\arrow[scale=4,>=stealth]{>}}},postaction={decorate}, line width=.75mm] (0,0) -- (20,0) node[right] {Rate};

                %\fill[yellow, fill opacity=0.4] (8,0) -- (8,7)  -- (15,0);
                \draw[loosely dashed, line width=1mm, blue, very thick] (8,7) -- (8,0);
                \draw[line width=1mm, red, very thick] (20,7) -- (0,7);

                \draw[loosely dashed, line width=1mm] (3,12) -- (0,12) node[left] {};
                \draw[loosely dashed, line width=1mm] (3,12) -- (3,0);
                \draw[|<->|, line width=1mm] (0,-0.75) -- node[fill=white] {$R(\hat{x},\hat{y})$} (3,-0.75);
                \draw[|<->|, line width=1mm] (0,-2) -- node[fill=white] {$\frac{\text{SAD}(\mathbb{S},0,\hat{x},\hat{y}) - \lowersad}{\lambda} + R(\hat{x},\hat{y})$} (8,-2);
                \draw[|<->|, line width=1mm] (0,-3) -- node[fill=white] {$\frac{\text{SAD}(\mathbb{S},0,\hat{x},\hat{y})}{\lambda} + R(\hat{x},\hat{y})$} (15,-3);

                \draw[|<->|, line width=1mm] (-1,0) -- node[fill=white, rotate=90] {$\lowersad$} (-1,7);
                \draw[|<->|, line width=1mm] (-2,0) -- node[fill=white, rotate=90] {$\text{SAD}(\mathbb{S},0,\hat{x},\hat{y})$} (-2,12);
                \draw[|<->|, line width=1mm] (-3,0) -- node[fill=white, rotate=90] {$\text{SAD}(\mathbb{S},0,\hat{x}, \hat{y}) + \lambda R(\hat{x}, \hat{y})$} (-3,15);

                \draw[->, line width=1mm, blue] (10, 11) node[above, align=center] {\textbf{Proposed Early Termination Threshold}} to[out=270,in=45] (8,7.1);
                \draw[->, line width=1mm, text width=6cm] (14, 4) node[above, align=center] {Previous Early Termination Threshold} to[out=270,in=45] (15,0.1);

                \draw[->, line width=1mm, text width=6cm] (19, 4) node[above] {Edge of the search area} to[out=270,in=90] (19,0.1);

                \draw[->, line width=1mm, text width=8cm, red] (14, 9) node[above, align=center] {\textbf{New: No candidates below this line}} to[out=270,in=90] (14,7.1);


                \fill[yellow, fill opacity=0.2] (0,16) -- (0,15) -- (8,7) -- (19,7) -- (19,16);
                \node[text width=15cm, orange] at (14, 15) {Candidates are only in this region};

                \end{tikzpicture}
                \vspace{-0.5em}
                \caption{Geometric representation of the early termination thresholds.}
                \label{fig:Triangle}
                \end{figure}
        \end{block}


                    %%%%%%%%%%%%%%%%%%%%%%%%% EXPERIMENTAL RESULTS %%%%%%%%%%%%%%%%%%%%%%%%%
        \begin{block}{\titlecap{5. Experimental Results}}
            \begin{table}
\centering
\small{
\begin{tabular}{|c|l|r|r|r||r|r|r|}
 \cline{3-8} \multicolumn{2}{c|}{} & \multicolumn{3}{c||}{\textbf{Prop. vs HM}} & \multicolumn{3}{c|}{\textbf{Prop. vs Previous (ICIP2014)}} \\ \hline
 Class & Sequence name  & \multicolumn{1}{c|}{Speedup} & \multicolumn{1}{c|}{SAD} & \multicolumn{1}{c||}{BD-PSNR} & \multicolumn{1}{c|}{Speedup} & \multicolumn{1}{c|}{SAD} & \multicolumn{1}{c|}{$\mathbb{S}$ SAD} \\
 &  & \multicolumn{1}{c|}{} & \multicolumn{1}{c|}{Savings} & & \multicolumn{1}{c|}{} & \multicolumn{1}{c|}{Savings} & \multicolumn{1}{c|}{Savings} \\ \hline
 \multicolumn{1}{|c|}{\multirow{5}{5.85cm}{\hfil\quad~B\\\hfil$(1920\times1080)$}}  & Kimono & 6.30 & 96.7\% & 0.0006 & 1.15  & 14.9\% & 45.6\% \\
& ParkScene & 6.42 & 95.8\% & 0.0014 & 1.35 & 25.7\% & 79.4\% \\
& Cactus & 7.07 & 96.3\% & 0.0018 & 1.27 & 21.8\% & 67.9\% \\
& BQTerrace & 5.92 & 94.6\% & -0.0020 & 1.36 & 26.3\% & 81.9\% \\
& BasketballDrive & 6.05 & 95.4\% & 0.0016 & 1.23 & 20.2\% & 64.0\% \\ \hline
 \multicolumn{1}{|c|}{\multirow{4}{5.85cm}{\hfil\quad~C\\\hfil$(832\times480)$}}  & RaceHorses C & 4.73 & 92.7\% & 0.0011 & 1.13 & 14.8\% & 50.2\% \\
& BQMall & 6.70 & 95.5\% & -0.0008 & 1.18 & 16.0\% & 53.1\% \\
& PartyScene & 4.68 & 91.6\% & -0.0003 & 1.27 & 19.9\% & 66.2\% \\
& BasketballDrill & 5.59 & 95.4\% & -0.0026 & 1.24 & 19.3\% & 61.0\%   \\ \hline
 \multicolumn{1}{|c|}{\multirow{4}{5.85cm}{\hfil\quad~D\\\hfil$(416\times240)$}}  & RaceHorses & 4.56 & 93.0\% & -0.0030 & 1.15 & 12.9\% & 43.1\% \\
& BQSquare & 8.75 & 96.1\%  & 0.0032 & 1.34 & 27.6\% & 90.4\% \\
& BlowingBubbles & 6.78 & 95.2\%  & -0.0020  & 1.22 & 20.7\% & 68.1\% \\
& BasketballPass & 6.18 & 95.4\%  & -0.0011 & 1.20 & 17.8\% & 56.9\% \\ \hline
\multicolumn{1}{r|}{} & \textbf{Overall} & 6.13 & 94.9\% & 0.0002 & 1.23 & 19.8\% & 63.7\% \\  \cline{2-8}
\end{tabular}}
\caption{Results for main profile with 8-bit coding and Low Delay P settings (No AMP)}
\label{table:LDSummary}
\vspace{-0.5em}
\end{table}
            \begin{figure}[htb]
                \centering
                % Color map from based on http://colorbrewer2.org/
%  * colorblind safe
%  * print friendly
%  * for sequential data
\definecolor{c1}{rgb}{0.9961, 0.9412, 0.8510}%
\definecolor{c2}{rgb}{0.9922, 0.8000, 0.5412}%
\definecolor{c3}{rgb}{0.9882, 0.5529, 0.3490}%
\definecolor{c4}{rgb}{0.8431, 0.1882, 0.1216}%

\begin{tikzpicture}

\begin{axis}[%
width=30cm,
height=22cm,
at={(0in,0in)},
scale only axis,
xmin=0,
xmax=100,
xlabel={\small{\% of $\mathbb{S}$ SAD Savings over previous threshold adapted to HEVC}},
xlabel style={yshift=-0.5em},
every x tick label/.append style={font=\color{black}\footnotesize},
ymin=0,
ymax=15,
ytick={1,2,3,4,5,6,7,8,9,10,11,12,13,14},
yticklabels={{Overall},{BasketballPass},{BlowingBubbles},{BQSquare},{RaceHorses},{BasketballDrill},{PartyScene},{BQMall},{RaceHorses C},{BasketballDrive},{BQTerrace},{Cactus},{ParkScene},{Kimono}},
axis background/.style={fill=white, font=\footnotesize},
every y tick label/.append style={font=\color{black}\footnotesize},
legend style={at={(0.45,1.03)},anchor=south,legend columns=6,legend cell align=left,align=left,draw=white!15!black,font=\footnotesize,
reverse legend}
]

\addplot[xbar,bar width=3mm,bar shift=-4.5mm,draw=black,fill=c4,line width=0.25mm,area legend] plot table[row sep=crcr] {%
70.8	1\\
66.9	2\\
77.5	3\\
93.6	4\\
54.8	5\\
67.0	6\\
76.8	7\\
62.5	8\\
60.5	9\\
67.4	10\\
88.4	11\\
72.1	12\\
84.2	13\\
48.9	14\\
};
\addlegendentry{37};

\addplot[xbar,bar width=3mm,bar shift=-1.5mm,draw=black,fill=c3,line width=0.25mm,area legend] plot table[row sep=crcr]
{%
66.2	1\\
60.1	2\\
71.8	3\\
92.4	4\\
47.4	5\\
63.4	6\\
69.9	7\\
55.8	8\\
52.9	9\\
65.5	10\\
85.7	11\\
68.7	12\\
81.1	13\\
46.2	14\\
};
\addlegendentry{32};

\addplot[xbar,bar width=3mm,bar shift=1.5mm,draw=black,fill=c2,line width=0.25mm,area legend] plot table[row sep=crcr]
{%
61.2	1\\
53.2	2\\
65	3\\
90.1	4\\
38.3	5\\
59.6	6\\
62.3	7\\
49.3	8\\
46.1	9\\
63.3	10\\
81.1	11\\
65.4	12\\
77.6	13\\
44.8	14\\
};
\addlegendentry{27};

\addplot[xbar,bar width=3mm,bar shift=4.5mm,draw=black,fill=c1,line width=0.25mm,area legend] plot table[row sep=crcr] {%
56.5	1\\
47.4	2\\
58.1	3\\
85.6	4\\
31.8	5\\
54.1	6\\
55.7	7\\
44.9	8\\
41.5	9\\
59.8	10\\
72.5	11\\
65.4	12\\
74.9	13\\
42.4	14\\
};
\addlegendentry{22};

\addlegendimage{empty legend}
\addlegendentry{\hspace{-.1cm}\textbf{QP:}}

\end{axis}
\draw[line width=0.02mm, black, dashed] (-6,2.2) -- (30,2.2);
\draw[line width=0.05mm, black, dashed] (-6,8.05) -- (30,8.05);
\draw[line width=0.05mm, black, dashed] (-6,13.95) -- (30,13.95);
\end{tikzpicture}%
                \vspace{-0.5em}
                \caption{Results for main profile with 8-bit coding and Low Delay P settings (No AMP)}
                \label{fig:SADSavingBar}
            \end{figure}
        \end{block}

                %%%%%%%%%%%%%%%%%%%%%%%%% CONCLUSION %%%%%%%%%%%%%%%%%%%%%%%%%
        \begin{block}{\titlecap{6. Conclusion}}
        \begin{itemize}
            \item The proposed early termination scheme for square \glspl{pu}, based on information reuse from rectangular \glspl{pu}, results in a \textbf{6.13x} speedup and \textbf{94.9\%} SAD savings when compared to HM (Full Search).
            \item This work considerably decreases the number of candidates imposed by the HEVC standard in order to find the optimal solution.
        \end{itemize}
        \end{block}

    \end{column}
\end{columns}
\end{frame}
\end{document}
